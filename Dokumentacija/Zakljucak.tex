\chapter{Zaključak i budući rad}
		
		 
		 Zadatak naše grupe bio je napraviti web aplikaciju čija je svrha ubrzati digitalizaciju računovodstvenim tvrtkama. Jedna od glavnih funkcionalnosti aplikacije je detekcija dokumenta na slici te izrada OCR-a detektiranog teksta. Kroz 15 tjedana koliko je trajao rad na projektu, aplikacija je u pogonu, a sama provedba projekta imala je dvije faze - fazu u prvom ciklusu predavanja i fazu u drugom ciklusu predavanja.
		 
		 Prva faza počela je tako da smo se okupili u timu te nam je asistent detaljno objasnio zadatak i dao upute kako će cijeli ovaj projekt funkcionirati. Mi smo u timu odlučili da ćemo koristiti Javu kao programski jezik i MyPostgreSQL, tj. pgAdmin za bazu podataka. Nakon toga, krenulo je pisanje dokumentacije, smišljanje funkcionalnosti, osmišljavanje strukture baze podataka te crtanje raznih dijagrama. Krajnji cilj aplikacije u prvoj fazi bio je napraviti samo stranicu za prijavu i registraciju što smo uspješno napravili u zadanom roku.
		 
		 Druga faza bila je vremenski kraća od prve, ali u njoj je bilo puno više posla. Sada nam je cilj bio implementirati sve funkcionalnosti koje smo smislili u prvoj fazi. Članovi tima nisu imali previše iskustva u radu aplikacije, posebno s backend dijelom. Frontend dio smo pisali u HMTL-u i CSS-u s kojima smo se susreli prošle godine na predmetu Razvoj programske potpore za web tako da nam je veći problem radio backend dio. Isto tako, uz samu implementaciju trebalo je popraviti i dokumentaciju te ju prilagoditi našoj aplikaciji. Naime, dio dokumentacije iz prve faze je više bilo nagađanje što želimo imati tako da je dokumentacija trebala biti uređena. Uz uređivanje i prilagodbu, morali smo nacrtati preostale dijagrame - dijagram komponenti, dijagram razmještaja, dijagram stanja, dijagram aktivnosti te konačan dijagram razreda što smo uspješno i napravili.
		 
		 Što se tiče same aplikacije, zadovoljni smo napravljenim, ali isto tako znamo da je to sve moglo biti sve skupa bolje. Nismo uspjeli implementirati notifikacije među zaposlenicima prilikom slanja dokumenata nego se stranica mora osvježiti kako bi se vidjela poruka. Dakle, ukoliko netko dobije poruku, neće vidjeti da ju je dobio dok ne osvježi stranicu. Cilj je bio napraviti da se odmah dobije obavijest, ali nismo imali dovoljno znanja da to napravimo. Uz to, nezadovoljni smo sa svojim OCR-om koji smo skinuli s interneta i nismo ga znali poboljšati.
		 
		 Kroz ovaj projekt dobili smo iskustvo i upoznali smo se s radom u grupi s više članova. Svatko je ovisio o svakome, nije nitko bio sam za sebe što je velika novost u odnosu na sve što smo radili prije na fakultetu. Kako nismo imali znanja o tome kako se radi aplikacija i kako sve to funkcionira, dosta smo se mučili s implementacijom i zato je rad na projektu trajao jako dugo. Mislim da bi nam jako pomoglo da smo se prije susreli s radom na aplikaciji ili općenito s bilo kakim projektom te da bi u tom slučaju sve teklo brže i bolje.
		 
		 Aplikacija se, naravno, može puno poboljšati, ali s obzirom na to da smo imali manjak znanja i da nam je ovo bio prvi rad na aplikaciji, zadovoljni smo postignutim rezultatima te se nadamo da ćemo u nekim budućim projektima biti uspješniji..
		
		\eject 