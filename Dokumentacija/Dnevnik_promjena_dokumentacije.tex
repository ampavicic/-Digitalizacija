\chapter{Dnevnik promjena dokumentacije}
		
		\textbf{\textit{Kontinuirano osvježavanje}}\\
				
		
		\begin{longtblr}[
				label=none
			]{
				width = \textwidth, 
				colspec={|X[2]|X[13]|X[13]|X[7]|}, 
				rowhead = 1
			}
			\hline
			\textbf{Rev.}	& \textbf{Opis promjene/dodatka} & \textbf{Autori} & \textbf{Datum}\\[3pt] \hline
			0.1 & Napravljen predložak.	& Matej Lopotar & 29.10.2021. 		\\[3pt] \hline 
			0.2	& Opis projektnog zadatka. & Antonio Babić i Matej Lopotar & 31.10.2021. 	\\[3pt] \hline 
			0.3 & Napisani UC-ovi & Iwan Ćulemović, Josip Hanak \newline Ana Marija Pavičić & 7.11.2021. \\[3pt] \hline 
			0.4 & ER i REL sheme & Matej Lopotar i Iwan Ćulemović & 9.11.2021. \\[3pt] \hline 
			0.5 & Sekvencijski dijagrami & Matej Lopotar, Andrej Pogačić, Antonio Kuran & 16.11.2021. \\[3pt] \hline 
			0.6 & Dodane slike u dokumentaciju  & Matej Lopotar & 16.11.2021. \\[3pt] \hline 
			0.7 &Ispravak UC-ova  & Matej Lopotar & 16.11.2021. \\[3pt] \hline 
			0.8 & Arhitektura sustava, Dijagram razreda, Obrasci uporabe & Matej Lopotar i Antonio Babić & 19.11.2021. \\[3pt] \hline  
			1.1 & Uređivanje teksta -- funkcionalni i nefunkcionalni zahtjevi & * \newline * & 14.09.2013. \\[3pt] \hline 
			1.2 & Manje izmjene:Timer - Brojilo vremena & * & 15.09.2013. \\[3pt] \hline 
			1.3 & Popravljeni dijagrami obrazaca uporabe & * & 15.09.2013. \\[3pt] \hline 
			1.5 & Generalna revizija strukture dokumenta & * & 19.09.2013. \\[3pt] \hline 
			1.5.1 & Manja revizija (dijagram razmještaja) & * & 20.09.2013. \\[3pt] \hline 
			\textbf{2.0} & Konačni tekst predloška dokumentacije  & * & 28.09.2013. \\[3pt] \hline 
			&  &  & \\[3pt] \hline	
		\end{longtblr}
	
	
		\textit{Moraju postojati glavne revizije dokumenata 1.0 i 2.0 na kraju prvog i drugog ciklusa. Između tih revizija mogu postojati manje revizije već prema tome kako se dokument bude nadopunjavao. Očekuje se da nakon svake značajnije promjene (dodatka, izmjene, uklanjanja dijelova teksta i popratnih grafičkih sadržaja) dokumenta se to zabilježi kao revizija. Npr., revizije unutar prvog ciklusa će imati oznake 0.1, 0.2, …, 0.9, 0.10, 0.11.. sve do konačne revizije prvog ciklusa 1.0. U drugom ciklusu se nastavlja s revizijama 1.1, 1.2, itd.}