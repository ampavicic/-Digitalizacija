\chapter{Specifikacija programske potpore}
		
	\section{Funkcionalni zahtjevi}
			
			\textbf{\textit{dio 1. revizije}}\\
			
			\textit{Navesti \textbf{dionike} koji imaju \textbf{interes u ovom sustavu} ili  \textbf{su nositelji odgovornosti}. To su prije svega korisnici, ali i administratori sustava, naručitelji, razvojni tim.}\\
				
			\textit{Navesti \textbf{aktore} koji izravno \textbf{koriste} ili \textbf{komuniciraju sa sustavom}. Oni mogu imati inicijatorsku ulogu, tj. započinju određene procese u sustavu ili samo sudioničku ulogu, tj. obavljaju određeni posao. Za svakog aktora navesti funkcionalne zahtjeve koji se na njega odnose.}\\
			
			
			\noindent \textbf{Dionici:}
			
			\begin{packed_enum}
				
				\item Korisnici
				\item Admini sustava				
				\item Razvojni tim
				
			\end{packed_enum}
			
			\noindent \textbf{Aktori i njihovi funkcionalni zahtjevi:}
			
			
			\begin{packed_enum}
				\item  \underbar{Zaposlenik (inicijator) može:}
				
				\begin{packed_enum}
					
					\item učitavati slike
					\item skenirati dokumente
					\item pregledati povijest skeniranih dokumenata
					\item slati dokumente revizoru
					
				\end{packed_enum}
			
				\item  \underbar{Revizor (sudionik) može:}
				
				\begin{packed_enum}
					
					\item skenirati dokument
					\item provjeriti jeli dokumenat ispravan i poslati ga računovođi
					
				\end{packed_enum}
			
				\item  \underbar{Računovođa (sudionik) može:}
				
				\begin{packed_enum}
					
					\item dodjeljivati jedinstveni broj arhivu
					\item arhivirati dokument
					\item slati obavijest direktoru da se potpiše dokument
					
				\end{packed_enum}
			
				\item  \underbar{Direktor (sudionik) može:}
				
				\begin{packed_enum}
					
					\item potpisati dokumente i poslati obavijest da je dokument potpisan
					\item pregledati povijest svih dokumenata
					\item pregledati statistiku svih zaposlenika
					
				\end{packed_enum}
				
				\item  \underbar{Baza podataka (sudionik) može:}
				
				\begin{packed_enum}
					
					\item dodati nove dokumente i arhive u bazu
					\item vraćati povijest svih dokumenata
					\item vraćati statistiku svih zaposlenika
					
				\end{packed_enum}	
			
			\end{packed_enum}
			
			
			\eject 
			
			
				
			\subsection{Obrasci uporabe}
				
				\textbf{\textit{dio 1. revizije}}
				
				\subsubsection{Opis obrazaca uporabe}
					\textit{Funkcionalne zahtjeve razraditi u obliku obrazaca uporabe. Svaki obrazac je potrebno razraditi prema donjem predlošku. Ukoliko u nekom koraku može doći do odstupanja, potrebno je to odstupanje opisati i po mogućnosti ponuditi rješenje kojim bi se tijek obrasca vratio na osnovni tijek.}\\
					

					\noindent \underbar{\textbf{UC1 - Prilaganje dokumenta}}
					\begin{packed_item}
	
						\item \textbf{Glavni sudionik:} Zaposlenik
						\item  \textbf{Cilj:} Priložiti dokument
						\item  \textbf{Sudionici:} Revizor, baza podataka
						\item  \textbf{Preduvjet:} Ulogirani verificirani zaposlenik, funkcionalna kamera
						\item  \textbf{Opis osnovnog tijeka:}
						
						\item[] \begin{packed_enum}
	
							\item Zaposlenik klikće na označeno mjesto za prilaganje slika
							\item Zaposlenik označuje slike koje će se priložiti i prilaže ih
							\item Zaposlenik inicira OCR test
							\item Web aplikacija vraća dokumente u skeniranom obliku
							\item Zaposlenik označava jeli dokument krivo ili točno poslan
							\item Dokumenti se spremaju u bazu podataka (bili krivi ili točni) i šalju se na obradu revizoru
						\end{packed_enum}
						
						\item  \textbf{Opis mogućih odstupanja:}
						
						\item[] \begin{packed_item}
	
							\item[2.a] Korisnik je priložio više od 50 slika
							\item[] \begin{packed_enum}
								
								\item Javiti korisniku grešku i onemogućiti slanje
								
							\end{packed_enum}
							
						\end{packed_item}
					\end{packed_item}
				
					\noindent \underbar{\textbf{UC2 - Pregled povijesti skeniranih dokumenata}}
					\begin{packed_item}
						
						\item \textbf{Glavni sudionik:} Zaposlenik
						\item  \textbf{Cilj:} Dobiti listu povijsesti skeniranih dokumenata
						\item  \textbf{Sudionici:} Baza podataka
						\item  \textbf{Preduvjet:} Ulogirani verificirani zaposlenik
						\item  \textbf{Opis osnovnog tijeka:}
						
						\item[] \begin{packed_enum}
							
							\item Zaposlenik klikće na mjesto za prikaz povijesti dokumenata
							\item Web aplikacija šalje upit bazi podataka koji sadrži podatke o zaposleniku
							\item Baza podataka vraća tablicu dokumenata
							
						\end{packed_enum}
						
					\end{packed_item}
				
				\noindent \underbar{\textbf{UC3 - Verifikacija dokumenata}}
				\begin{packed_item}
					
					\item \textbf{Glavni sudionik:} Revizor
					\item  \textbf{Cilj:} Verificirati dokumente i poslati računovođi
					\item  \textbf{Sudionici:} Računovođa
					\item  \textbf{Preduvjet:} Zaposlenik mora nešto poslati
					\item  \textbf{Opis osnovnog tijeka:}
					
					\item[] \begin{packed_enum}
						
						\item Revizor dobiva obavijest u inboxu i poslani skenirani dokument
						\item Revizor provjerava dokument i sam ga šalje računovođi ili obavlja sken koji automatski šalje određenom računovođi

					\end{packed_enum}
					
				\end{packed_item}
					
				\noindent \underbar{\textbf{UC4 - Arhiviranje dokumenata}}
				\begin{packed_item}
					
					\item \textbf{Glavni sudionik:} Računovođa
					\item  \textbf{Cilj:} Dodjeliti jedinstveni broj arhiva i arhivirati dokument
					\item  \textbf{Sudionici:} Baza podataka
					\item  \textbf{Preduvjet:} Revizor mora nešto poslati
					\item  \textbf{Opis osnovnog tijeka:}
					
					\item[] \begin{packed_enum}
						
						\item Dodjeli jedinstveni broj arhiva dokumentu
						\item Spremi dokumente u bazu
						
					\end{packed_enum}
					
				\end{packed_item}
			
				\noindent \underbar{\textbf{UC5 - Direktorski potpis}}
				\begin{packed_item}
					
					\item \textbf{Glavni sudionik:} Direktor
					\item  \textbf{Cilj:} Potpisati dokument elektroničkim potpisom
					\item  \textbf{Sudionici:} Računovođa, baza podataka
					\item  \textbf{Preduvjet:} Računovođa mora poslati nearhivirane dokumente za potpis
					\item  \textbf{Opis osnovnog tijeka:}
					
					\item[] \begin{packed_enum}
						
						\item Direktor dobiva obavijest u inboxu i poslani nearhivirani dokument
						\item Direktor potpisuje dokument
						\item Direktor šalje obavijest da je dokument potpisan i sprema ga u bazu
						
					\end{packed_enum}
					
				\end{packed_item}
			
				\noindent \underbar{\textbf{UC6 - Pregled potpisanih dokumenata}}
				\begin{packed_item}
					
					\item \textbf{Glavni sudionik:} Direktor
					\item  \textbf{Cilj:} Dohvatiti listu svih potpisanih dokumenata
					\item  \textbf{Sudionici:} Baza podataka
					\item  \textbf{Preduvjet:} -
					\item  \textbf{Opis osnovnog tijeka:}
					
					\item[] \begin{packed_enum}
						
						\item Direktor šalje upit bazi podataka za tablicu potpisanih dokumenata 
						\item Baza vraća tablicu potpisanih dokumenata
						
					\end{packed_enum}
					
				\end{packed_item}
					
				\noindent \underbar{\textbf{UC7 - Pregled statistike}}
				\begin{packed_item}
					
					\item \textbf{Glavni sudionik:} Direktor
					\item  \textbf{Cilj:} Dohvatiti podatke o zaposlenicima
					\item  \textbf{Sudionici:} Baza podataka
					\item  \textbf{Preduvjet:} -
					\item  \textbf{Opis osnovnog tijeka:}
					
					\item[] \begin{packed_enum}
						
						\item Direktor šalje upit bazi podataka za statistiku o zaposlenicima
						\item Baza vraća tablicu zaposlenika
						
					\end{packed_enum}
					
				\end{packed_item}
			
				\noindent \underbar{\textbf{UC8 - prijava korisnika  }}
				\begin{packed_item}
					
					\item \textbf{Glavni sudionik:} Korisnik
					\item  \textbf{Cilj:} Omogućiti prijavu postojećim korisnicima ili odabrati registraciju novih
					\item  \textbf{Sudionici:} Baza podataka
					\item  \textbf{Preduvjet:} U bazi podataka mora postojati račun korisnika, odnosno direktor mora odobriti registraciju
					\item  \textbf{Opis osnovnog tijeka:}
					
					\item[] \begin{packed_enum}
						
						\item Korisnik unosi svoje korisničke detalje i nakon provjere ulazi u home page
						\item Nepostojeći korisnici imaju opciju preusmjeravanja na registraciju
						
					\end{packed_enum}
					\item  \textbf{Opis mogućih odstupanja:}
					
					\item[] \begin{packed_item}
						
						\item[2.a] Korisnik je unio korisničke detalje su krivi ili nepostojani
						\item[] \begin{packed_enum}
							
							\item Ako je korisničko ime poznato ali korisnička šifra kriva korisnik mora ponovno unijeti šifru dok ne bude ispravna
							\item Ako je korisničko ime nepostojano unutar baze podataka korisnika se navodi na registraciju
						\end{packed_enum}
						
					\end{packed_item}					
				\end{packed_item}
				
				\noindent \underbar{\textbf{UC9 - Registracija korisnika}}
				\begin{packed_item}
					
					\item \textbf{Glavni sudionik:} Korisnik
					\item  \textbf{Cilj:} Registrirati nove članove
					\item  \textbf{Sudionici:} Baza podataka, Direktor
					\item  \textbf{Preduvjet:} 
					\item  \textbf{Opis osnovnog tijeka:}
					
					\item[] \begin{packed_enum}
						
						\item Korisnik odabire opciju registracije te unosi svoje osnovne podatke i poziciju
						
					\end{packed_enum}
					
				\end{packed_item}
				
				\noindent \underbar{\textbf{UC10 - Prilaganje dokumenta}}
				\begin{packed_item}
					
					\item \textbf{Glavni sudionik:} Korisnik
					\item  \textbf{Cilj:} Priložiti dokument
					\item  \textbf{Sudionici:} Baze podataka
					\item  \textbf{Preduvjet:} Ulogirani verificirani korisnik, funkcionalna kamera, ispravno skenirani dokument
					\item  \textbf{Opis osnovnog tijeka:}
					
					\item[] \begin{packed_enum}
						
						\item Korisnik odabire  označeno mjesto za prilaganje slika
						\item Korisnik označuje slike koje će se priložiti i prilaže ih
						\item Korisnik inicira OCR testiranje
						\item Web aplikacija vraća dokumente u skeniranom obliku
						\item Korisnik označava je li dokument krivo ili točno poslan
						\item Dokumenti se spremaju u bazu podataka 						\end{packed_enum}
					
					\item  \textbf{Opis mogućih odstupanja:}
					
					\item[] \begin{packed_item}
						
						\item[2.a] Korisnik je priložio više od 50 slika
						\item[] \begin{packed_enum}
							
							\item Javiti korisniku grešku i onemogućiti slanje
							\item OCR je krivo parsirao dokument
						\end{packed_enum}
						
					\end{packed_item}
				\end{packed_item}
				
				
				\noindent \underbar{\textbf{UC11 - Pregled priloženih dokumenata}}
				\begin{packed_item}
					
					\item \textbf{Glavni sudionik:} Korisnik
					\item  \textbf{Cilj:} Omogućiti pregled svih dokumenata koje je korisnik priložio
					\item  \textbf{Sudionici:} Baze podataka
					\item  \textbf{Preduvjet:} Korisnik mora imati priložene dokumente
					\item  \textbf{Opis osnovnog tijeka:}
					
					\item[] \begin{packed_enum}
						
						\item Korisniku odabire pregled priloženih dokumenata
						\item Korisiniku su prikazani svi dokumenti te može odabrati pojedinačno dokument 
						
					\end{packed_enum}
					
				\end{packed_item}
				
				\noindent \underbar{\textbf{UC12 - Dodavanje sažetka dokumentu}}
				\begin{packed_item}
					
					\item \textbf{Glavni sudionik:} Zaposlenik
					\item  \textbf{Cilj:} Dodavanje kratkog opisa svakom priloženom dokumentu
					\item  \textbf{Sudionici:} Baza podataka
					\item  \textbf{Preduvjet:} 
					\item  \textbf{Opis osnovnog tijeka:}
					
					\item[] \begin{packed_enum}
						
						\item Korisnik odabire dokument te ima opciju nadodati kratak opis cijelog dokumenta
						
					\end{packed_enum}
					
				\end{packed_item}
				
				\noindent \underbar{\textbf{UC13 - Brisanje dokumenata}}
				\begin{packed_item}
					
					\item \textbf{Glavni sudionik:} Korisnik
					\item  \textbf{Cilj:} Omogućiti brisanje dokumenata iz baze
					\item  \textbf{Sudionici:} Baza podataka
					\item  \textbf{Preduvjet:} 
					\item  \textbf{Opis osnovnog tijeka:}
					
					\item[] \begin{packed_enum}
						
						\item Korisnik odabire dokument
						\item Korisnik odabire opciju obriši
						
					\end{packed_enum}
					
				\end{packed_item}
				
				
				\noindent \underbar{\textbf{UC14 - Slanje dokumenata}}
				\begin{packed_item}
					
					\item \textbf{Glavni sudionik:} Revizor, računovođa, direktor
					\item  \textbf{Cilj:} Slanje dokumenata na daljnju obradu
					\item  \textbf{Sudionici:} 
					\item  \textbf{Preduvjet:} Ako revizor šalje računovođi dokument mora biti verificiran
					\item  \textbf{Opis osnovnog tijeka:}
					
					\item[] \begin{packed_enum}
						
						\item Korisnik odabire dokumente koje želi poslati te odabire jednog od 									ponuđenih korisnika
						
					\end{packed_enum}
					
				\end{packed_item}
				
				
				\noindent \underbar{\textbf{UC15 -	Pregled dobivenih dokumenata}}
				\begin{packed_item}
					
					\item \textbf{Glavni sudionik:} Revizor, računovođa, direktor
					\item  \textbf{Cilj:} Omogućiti pregled dobivenih dokumenata
					\item  \textbf{Sudionici:} 
					\item  \textbf{Preduvjet:} 
					\item  \textbf{Opis osnovnog tijeka:}
					
					\item[] \begin{packed_enum}
						
						\item Korisnik odabire pregled dobivenih dokumenata te može pojedinačno pregledati dokumente te daljnje postupati prema svojoj funkciji
						
					\end{packed_enum}
					
				\end{packed_item}
				
				
				\noindent \underbar{\textbf{UC16 - Verifikacija dokumenata}}
				\begin{packed_item}
					
					\item \textbf{Glavni sudionik:} Revizor
					\item  \textbf{Cilj:} Verificirati dokumente i poslati računovođi
					\item  \textbf{Sudionici:} 
					\item  \textbf{Preduvjet:} Svi dokumenti moraju biti verificirani
					\item  \textbf{Opis osnovnog tijeka:}
					
					\item[] \begin{packed_enum}
						
						\item Revizor dobiva obavijest u inboxu i poslani skenirani dokument
						\item Revizor provjerava dokument i sam ga šalje računovođi ili obavlja sken koji automatski šalje određenom računovođi
						
					\end{packed_enum}
					
				\end{packed_item}
				
				
				\subsubsection{Dijagrami obrazaca uporabe}
					
					\textit{Prikazati odnos aktora i obrazaca uporabe odgovarajućim UML dijagramom. Nije nužno nacrtati sve na jednom dijagramu. Modelirati po razinama apstrakcije i skupovima srodnih funkcionalnosti.}
				\eject		
				
			\subsection{Sekvencijski dijagrami}
				
				\textbf{\textit{dio 1. revizije}}\\
				
				\textit{Nacrtati sekvencijske dijagrame koji modeliraju najvažnije dijelove sustava (max. 4 dijagrama). Ukoliko postoji nedoumica oko odabira, razjasniti s asistentom. Uz svaki dijagram napisati detaljni opis dijagrama.}
				\eject
	
		\section{Ostali zahtjevi}
		
			\textbf{\textit{dio 1. revizije}}\\
		 
			 \textit{Nefunkcionalni zahtjevi i zahtjevi domene primjene dopunjuju funkcionalne zahtjeve. Oni opisuju \textbf{kako se sustav treba ponašati} i koja \textbf{ograničenja} treba poštivati (performanse, korisničko iskustvo, pouzdanost, standardi kvalitete, sigurnost...). Primjeri takvih zahtjeva u Vašem projektu mogu biti: podržani jezici korisničkog sučelja, vrijeme odziva, najveći mogući podržani broj korisnika, podržane web/mobilne platforme, razina zaštite (protokoli komunikacije, kriptiranje...)... Svaki takav zahtjev potrebno je navesti u jednoj ili dvije rečenice.}
			 
			 
			 
	