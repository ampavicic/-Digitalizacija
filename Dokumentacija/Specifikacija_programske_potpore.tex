\chapter{Specifikacija programske potpore}
		
	\section{Funkcionalni zahtjevi}
			
			\textbf{\textit{dio 1. revizije}}\\
			
			\textit{Navesti \textbf{dionike} koji imaju \textbf{interes u ovom sustavu} ili  \textbf{su nositelji odgovornosti}. To su prije svega korisnici, ali i administratori sustava, naručitelji, razvojni tim.}\\
				
			\textit{Navesti \textbf{aktore} koji izravno \textbf{koriste} ili \textbf{komuniciraju sa sustavom}. Oni mogu imati inicijatorsku ulogu, tj. započinju određene procese u sustavu ili samo sudioničku ulogu, tj. obavljaju određeni posao. Za svakog aktora navesti funkcionalne zahtjeve koji se na njega odnose.}\\
			
			
			\noindent \textbf{Dionici:}
			
			\begin{packed_enum}
				
				\item Korisnici
				\item Admini sustava				
				\item Razvojni tim
				
			\end{packed_enum}
			
			\noindent \textbf{Aktori i njihovi funkcionalni zahtjevi:}
			
			
			\begin{packed_enum}
				\item  \underbar{Zaposlenik (inicijator) može:}
				
				\begin{packed_enum}
					
					\item učitavati slike
					\item skenirati dokumente
					\item pregledati povijest skeniranih dokumenata
					\item slati dokumente revizoru
					
				\end{packed_enum}
			
				\item  \underbar{Revizor (sudionik) može:}
				
				\begin{packed_enum}
					
					\item skenirati dokument
					\item provjeriti jeli dokumenat ispravan i poslati ga računovođi
					
				\end{packed_enum}
			
				\item  \underbar{Računovođa (sudionik) može:}
				
				\begin{packed_enum}
					
					\item dodjeljivati jedinstveni broj arhivu
					\item arhivirati dokument
					\item slati obavijest direktoru da se potpiše dokument
					
				\end{packed_enum}
			
				\item  \underbar{Direktor (sudionik) može:}
				
				\begin{packed_enum}
					
					\item potpisati dokumente i poslati obavijest da je dokument potpisan
					\item pregledati povijest svih dokumenata
					\item pregledati statistiku svih zaposlenika
					
				\end{packed_enum}
				
				\item  \underbar{Baza podataka (sudionik) može:}
				
				\begin{packed_enum}
					
					\item dodati nove dokumente i arhive u bazu
					\item vraćati povijest svih dokumenata
					\item vraćati statistiku svih zaposlenika
					
				\end{packed_enum}	
			
			\end{packed_enum}
			
			
			\eject 
			
			
				
			\subsection{Obrasci uporabe}
				
				\textbf{\textit{dio 1. revizije}}
				
				\subsubsection{Opis obrazaca uporabe}
					\textit{Funkcionalne zahtjeve razraditi u obliku obrazaca uporabe. Svaki obrazac je potrebno razraditi prema donjem predlošku. Ukoliko u nekom koraku može doći do odstupanja, potrebno je to odstupanje opisati i po mogućnosti ponuditi rješenje kojim bi se tijek obrasca vratio na osnovni tijek.}\\
					

					\noindent \underbar{\textbf{UC1 - Prilaganje dokumenta}}
					\begin{packed_item}
	
						\item \textbf{Glavni sudionik:} Zaposlenik
						\item  \textbf{Cilj:} Priložiti dokument
						\item  \textbf{Sudionici:} Revizor
						\item  \textbf{Preduvjet:} Ulogirani verificirani zaposlenik, funkcionalna kamera
						\item  \textbf{Opis osnovnog tijeka:}
						
						\item[] \begin{packed_enum}
	
							\item Zaposlenik klikće na označeno mjesto za prilaganje slika
							\item Zaposlenik označuje slike koje će se priložiti i prilaže ih
							\item Zaposlenik inicira OCR test
							\item Web aplikacija vraća dokumente u skeniranom obliku
							\item Zaposlenik označava jeli dokument krivo ili točno skeniran
							\item Zaposlenik šalje revizoru dokument
							
							
						\end{packed_enum}
						
						\item  \textbf{Opis mogućih odstupanja:}
						
						\item[] \begin{packed_item}
	
							\item[1.] Korisnik je priložio više od 50 slika
							\item[] \begin{packed_enum}
								
								\item Javiti korisniku grešku i onemogućiti slanje
								
							\end{packed_enum}
							
						\end{packed_item}
					\end{packed_item}
				
					\noindent \underbar{\textbf{UC2 - Pregled povijesti skeniranih dokumenata}}
					\begin{packed_item}
						
						\item \textbf{Glavni sudionik:} Zaposlenik
						\item  \textbf{Cilj:} Dobiti listu povijsesti skeniranih dokumenata
						\item  \textbf{Sudionici:} Baza podataka
						\item  \textbf{Preduvjet:} Ulogirani verificirani zaposlenik
						\item  \textbf{Opis osnovnog tijeka:}
						
						\item[] \begin{packed_enum}
							
							\item Zaposlenik klikće na mjesto za prikaz povijesti dokumenata
							\item Web aplikacija šalje upit bazi podataka koji sadrži podatke o zaposleniku
							\item Baza podataka vraća listu skeniranih dokumenata
							
						\end{packed_enum}
						
					\end{packed_item}
				
				\noindent \underbar{\textbf{UC3 - Verifikacija dokumenata}}
				\begin{packed_item}
					
					\item \textbf{Glavni sudionik:} Revizor
					\item  \textbf{Cilj:} Verificirati dokumente i poslati računovođi
					\item  \textbf{Sudionici:} Računovođa
					\item  \textbf{Preduvjet:} Zaposlenik mora nešto poslati
					\item  \textbf{Opis osnovnog tijeka:}
					
					\item[] \begin{packed_enum}
						
						\item Revizor dobiva poslani skenirani dokument
						\item Revizor provjerava dokument i sam ga šalje ga web aplikaciji
						\item Web aplikacija prosljeđuje dokument računovođi kojeg je revizor odabrao
						\item Ako je revizor skenirao dokument, web aplikacija će sama odrediti kojem se računovođi šalje

					\end{packed_enum}
					
				\end{packed_item}
					
				\noindent \underbar{\textbf{UC4 - Arhiviranje dokumenata}}
				\begin{packed_item}
					
					\item \textbf{Glavni sudionik:} Računovođa
					\item  \textbf{Cilj:} Dodjeliti jedinstveni broj arhiva i arhivirati dokument
					\item  \textbf{Sudionici:} Baza podataka
					\item  \textbf{Preduvjet:} Revizor mora nešto poslati
					\item  \textbf{Opis osnovnog tijeka:}
					
					\item[] \begin{packed_enum}
						
						\item Računovođa šalje dokumente za arhivirat web aplikaciji
						\item Web aplikacija šalje bazi podataka dokumente koje je potrebno arhivirati
						
					\end{packed_enum}
					
				\end{packed_item}
			
				\noindent \underbar{\textbf{UC5 - Direktorski potpis}}
				\begin{packed_item}
					
					\item \textbf{Glavni sudionik:} Direktor
					\item  \textbf{Cilj:} Potpisati dokument elektroničkim potpisom
					\item  \textbf{Sudionici:} Računovođa, baza podataka
					\item  \textbf{Preduvjet:} Računovođa mora poslati nearhivirane dokumente za potpis
					\item  \textbf{Opis osnovnog tijeka:}
					
					\item[] \begin{packed_enum}
						
						\item Računovođa šalje web aplikaciji nepotpisani nearhivirani dokument
						\item Web aplikacija ga prosljeđuje direktoru
						\item Direktor dobiva obavijest u inboxu i poslani nearhivirani dokument
						\item Direktor potpisuje dokument i šalje potpisani dokument web aplikaciji
						\item Web aplikacija šalje upit za arhiviranje potpisanog dokumenta u bazi podataka 
						
					\end{packed_enum}
					
				\end{packed_item}
			
				\noindent \underbar{\textbf{UC6 - Pregled potpisanih dokumenata}}
				\begin{packed_item}
					
					\item \textbf{Glavni sudionik:} Direktor
					\item  \textbf{Cilj:} Dohvatiti listu svih potpisanih dokumenata
					\item  \textbf{Sudionici:} Baza podataka
					\item  \textbf{Preduvjet:} -
					\item  \textbf{Opis osnovnog tijeka:}
					
					\item[] \begin{packed_enum}
						
						\item Direktor šalje upit web aplikaciji za za potpisane dokumente 
						\item Web aplikacija šalje upit bazi podataka
						\item Nakon što baza podataka vrati listu dokumenata, web aplikacija je prosljeđuje direktoru
						\item Direktor odabrati iz liste potpisani dokument i pregledati ga
						
					\end{packed_enum}
					
				\end{packed_item}
					
				\noindent \underbar{\textbf{UC7 - Pregled podataka o zaposlenicima}}
				\begin{packed_item}
					
					\item \textbf{Glavni sudionik:} Direktor
					\item  \textbf{Cilj:} Dohvatiti podatke o zaposlenicima
					\item  \textbf{Sudionici:} Baza podataka
					\item  \textbf{Preduvjet:} -
					\item  \textbf{Opis osnovnog tijeka:}
					
					\item[] \begin{packed_enum}
						
						\item Direktor šalje upit web aplikaciji o statistici o zaposlenicima
						\item Web aplikacija šalje upit bazi podataka
						\item Nakon što baza podataka vrati listu, web aplikacija je prosljeđuje direktoru
						\item Direktor odabrati iz liste zaposlenika i pregledati sve o njemu
						
					\end{packed_enum}
					
				\end{packed_item}
			
				\noindent \underbar{\textbf{UC8 - Prijava korisnika}}
				\begin{packed_item}
					
					\item \textbf{Glavni sudionik:} Zaposlenik
					\item  \textbf{Cilj:} Omogućiti prijavu već registriranim zaposlenicima ili odabrati registraciju novih
					\item  \textbf{Sudionici:} Baza podataka
					\item  \textbf{Preduvjet:} U bazi podataka mora postojati račun zaposlenika
					\item  \textbf{Opis osnovnog tijeka:}
					
					\item[] \begin{packed_enum}
						
						\item Zaposlenik unosi svoje korisničke detalje i nakon provjere ulazi u početnu stranicu 
						\item Neregistrirani zaposlenici imaju opciju preusmjerenja na registraciju
						\item Ako je sve uspješno bilo, web aplikacija započinje sesiju
						
					\end{packed_enum}
					\item  \textbf{Opis mogućih odstupanja:}
					
					\item[] \begin{packed_item}
						
						\item[1.] Zaposlenik je unio korisničke detalje su krivi ili nepostojani
						\item[] \begin{packed_enum}
							
							\item Ako je korisničko ime poznato ali korisnička šifra kriva zaposlenik mora ponovno unijeti šifru dok ne bude ispravna
							\item Ako je korisničko ime nepostojano unutar baze podataka zaposlenika se navodi na registraciju
							
						\end{packed_enum}
						\item[2.] Ako je račun bio deaktiviran, ponovno se aktivira
						
						
					\end{packed_item}			
				\end{packed_item}
			
				\noindent \underbar{\textbf{UC9 - Odjava korisnika}}
				\begin{packed_item}
					
					\item \textbf{Glavni sudionik:} Zaposlenik
					\item  \textbf{Cilj:} Omogućiti odjavu zaposlenika
					\item  \textbf{Sudionici:} -
					\item  \textbf{Preduvjet:} -
					\item  \textbf{Opis osnovnog tijeka:}
					
					\item[] \begin{packed_enum}
						
						\item Zaposlenik odabire opciju odjave
						\item Web aplikacija završava sesiju
						
					\end{packed_enum}
					
				\end{packed_item}
				
				\noindent \underbar{\textbf{UC10 - Otvaranje korisničkog računa}}
				\begin{packed_item}
					
					\item \textbf{Glavni sudionik:} Zaposlenik
					\item  \textbf{Cilj:} Registrirati nove zaposlenike
					\item  \textbf{Sudionici:} Baza podataka
					\item  \textbf{Preduvjet:} -
					\item  \textbf{Opis osnovnog tijeka:}
					
					\item[] \begin{packed_enum}
						
						\item Korisnik odabire opciju registracije te unosi svoje osnovne podatke, jedinstvenu šifru što je dobio pri zaposlenju te aktivira svoj račun
						\item Web aplikacija dobiva obrazac te provjerava postoji li zaposlena osoba sa odgovarajućom jedinstvenom šifrom, imenom i prezimenom
						\item Ako postoji zaposlena osoba, u bazi podataka će se umetnuti novi korisnički račun 
						
					\end{packed_enum}
				
					\item  \textbf{Opis mogućih odstupanja:}
					
					\item[] \begin{packed_item}
						
						\item[1.] Zaposlenik je unio nepostojeću šifru u bazi podataka (ne postoji zaposlenik s tom šifrom)
						
						\item[] \begin{packed_enum}
							
							\item Odbiti registraciju
							
						\end{packed_enum}
						
					\end{packed_item}
					
				\end{packed_item}
				
				\noindent \underbar{\textbf{UC11 - Deaktivacija korisničkog računa}}
				\begin{packed_item}
					
					\item \textbf{Glavni sudionik:} Korisnik
					\item  \textbf{Cilj:} Deaktivirati račun korisnika
					\item  \textbf{Sudionici:} Baze podataka
					\item  \textbf{Preduvjet:} -
					\item  \textbf{Opis osnovnog tijeka:}
					
					\item[] \begin{packed_enum}
						
						\item Korisnik odabire opciju deaktivacije računa, potvrđuje deaktivaciju te upisuje zaporku
						\item Web aplikacija će dobiti obrazac te će deaktivirati korisnički račun slanjem upita bazi podataka za deaktivacijom korisničkog računa
						\item Baza podataka će postaviti atribut aktivnog računa na false
						
					\end{packed_enum}
					
				\end{packed_item}
			
				\noindent \underbar{\textbf{UC12 - Potpuno brisanje korisničkog računa}}
				\begin{packed_item}
					
					\item \textbf{Glavni sudionik:} Direktor
					\item  \textbf{Cilj:} Obrisati račun korisnika
					\item  \textbf{Sudionici:} Baze podataka
					\item  \textbf{Preduvjet:} Zaposlenik mora dobit otkaz
					\item  \textbf{Opis osnovnog tijeka:}
					
					\item[] \begin{packed_enum}
						
						\item Ako je zaposlenik dobio otkaz nema više pravo koristiti aplikaciju te baza podataka uklanja račune onima koji su dobili otkaz (ON DELETE)
						
					\end{packed_enum}
					
				\end{packed_item}
				
				\noindent \underbar{\textbf{UC13 - Dodavanje komenatara dokumentu}}
				\begin{packed_item}
					
					\item \textbf{Glavni sudionik:} Zaposlenik
					\item  \textbf{Cilj:} Dodavanje kratkog opisa svakom priloženom dokumentu
					\item  \textbf{Sudionici:} Baza podataka
					\item  \textbf{Preduvjet:} -
					\item  \textbf{Opis osnovnog tijeka:}
					
					\item[] \begin{packed_enum}
						
						\item Zaposlenik odabire dokument te ima opciju nadodati kratak opis cijelog dokumenta
						
					\end{packed_enum}
					
				\end{packed_item}
				
				\noindent \underbar{\textbf{UC14 - Pregled arhiviranih dokumenata}}
				\begin{packed_item}
					
					\item \textbf{Glavni sudionik:} Računovođa
					\item  \textbf{Cilj:} Omogućiti pregled svih arhiviranih dokumenata
					\item  \textbf{Sudionici:} Baza podataka
					\item  \textbf{Preduvjet:} -
					\item  \textbf{Opis osnovnog tijeka:}
					
					\item[] \begin{packed_enum}
						
						\item Računovođa bira pregled arhiviranih dokumenata
						\item Web aplikacija šalje upit bazi podataka za listu sa arhiviranim dokumentima
						\item Nakon što baza vrati listu, web aplikacija je prosljeđuje računovođi
						\item Računovođi su prikazani svi arhivirani dokumenti te ih može pojedinačno odabrati i pregledati
						
					\end{packed_enum}
					
				\end{packed_item}
				
				\noindent \underbar{\textbf{UC15 - Pregled hijerarhija zaposlenika}}
				\begin{packed_item}
					
					\item \textbf{Glavni sudionik:} Zaposlenik
					\item  \textbf{Cilj:} Omogućiti prikaz hijarhije zaposlenika unutar firme
					\item  \textbf{Sudionici:} Baza podataka
					\item  \textbf{Preduvjet:} -
					\item  \textbf{Opis osnovnog tijeka:}
					
					\item[] \begin{packed_enum}
						
						\item Zaposleniks odabire pregled hijerarhije
						\item Web aplikacija šalje upit bazi podataka za listu sa hijerarhijom zaposlenih
						\item Nakon što baza vrati listu, web aplikacija je prosljeđuje zaposleniku
						
					\end{packed_enum}
					
				\end{packed_item}
				
				\noindent \underbar{\textbf{UC16 - Pregled novijih registracija}}
				\begin{packed_item}
					
					\item \textbf{Glavni sudionik:} Direktor
					\item  \textbf{Cilj:} Omogućiti pregled novih registracija
					\item  \textbf{Sudionici:} Baza podataka
					\item  \textbf{Preduvjet:} -
					\item  \textbf{Opis osnovnog tijeka:}
					
					\item[] \begin{packed_enum}
						
						\item Direktor šalje upit web aplikaciji sa odabranim periodom zadnjih registracija npr. 1 dan, 1 tjedan, 3 mjeseca... (filtar)
						\item Web aplikacija šalje upit bazi podataka za listu sa uvjetom perioda
						\item Nakon što baza vrati listu, web aplikacija je prosljeđuje direktoru
						\item Direktor može pregledati listu registriranih u tom periodu
						
					\end{packed_enum}
					
				\end{packed_item}
				
				\noindent \underbar{\textbf{UC17 - Dodjela pozicije zaposleniku}}
				\begin{packed_item}
					
					\item \textbf{Glavni sudionik:} Direktor
					\item  \textbf{Cilj:} Dodjeliti poziciju zaposlenika u tvrtci
					\item  \textbf{Sudionici:} Baza podataka
					\item  \textbf{Preduvjet:} -
					\item  \textbf{Opis osnovnog tijeka:}
					
					\item[] \begin{packed_enum}
						
						\item Direktor šalje upit web aplikaciji sa ili šifrom ili imenom ili prezimenom zaposlenika (filtar)
						\item Web aplikacija šalje upit bazi podataka za listu sa ili šifrom ili imenom ili prezimenom 
						\item Nakon što baza vrati listu, web aplikacija je prosljeđuje direktoru
						\item Direktor iz liste može odabrati traženog zaposlenika i kliknuti na mjesto postavljanje uloge zaposlenika
						\item Šalje se uputa web aplikaciji za ažuriranje uloge zaposlenika specifične jedinstvene šifre
						
					\end{packed_enum}
					
				\end{packed_item}
			
				\noindent \underbar{\textbf{UC18 - Dodati zaposlenika u tablicu zaposlenih}}
				\begin{packed_item}
					
					\item \textbf{Glavni sudionik:} Direktor
					\item  \textbf{Cilj:} Zaposliti osobu
					\item  \textbf{Sudionici:} Baza podataka
					\item  \textbf{Preduvjet:} -
					\item  \textbf{Opis osnovnog tijeka:}
					
					\item[] \begin{packed_enum}
						
						\item Direktor upisuje sve podatke o osobi u zato predviđena mjesta te šalje obrazac web aplikaciji
						\item Web aplikacija generira jedinstvenu šifru te sve zajedno sprema u bazu podataka, a također još tu šifru vraća direktoru
						
					\end{packed_enum}
					
				\end{packed_item}
			
				\noindent \underbar{\textbf{UC19 - Ukloniti zaposlenika iz tablice zaposlenih}}
				\begin{packed_item}
					
					\item \textbf{Glavni sudionik:} Direktor
					\item  \textbf{Cilj:} Dati otkaz zaposlenom
					\item  \textbf{Sudionici:} Baza podataka
					\item  \textbf{Preduvjet:} -
					\item  \textbf{Opis osnovnog tijeka:}
					
					\item[] \begin{packed_enum}
						
						\item Direktor šalje upit web aplikaciji sa ili šifrom ili imenom ili prezimenom zaposlenika (filtar)
						\item Web aplikacija šalje upit bazi podatak za listu sa ili šifrom ili imenom ili prezimenom 
						\item Nakon što baza vrati listu, web aplikacija je prosljeđuje direktoru
						\item Direktor iz liste može odabrati traženog zaposlenika i kliknuti na gumb za davanje otkaza
						\item Šalje se uputa web aplikaciji za uklanjanje zaposlenika specifične jedinstvene šifre te web aplikacija uklanja zaposlenika te se njegova jedinstvena šifra deaktivira (također se miče iz baze podataka)
						\item račun zaposlenika (ako postoji) se također automatski uklanja iz baze podataka
								
					\end{packed_enum}
					
				\end{packed_item}

				\noindent \underbar{\textbf{UC20 - Postavljanje plaće zaposlenicima}}				
				\begin{packed_item}
					
					\item \textbf{Glavni sudionik:} Direktor
					\item  \textbf{Cilj:} Dati otkaz radniku
					\item  \textbf{Sudionici:} Baza podataka
					\item  \textbf{Preduvjet:} -
					\item  \textbf{Opis osnovnog tijeka:}
					
					\item[] \begin{packed_enum}
						
						\item Direktor šalje upit web aplikaciji sa ili šifrom ili imenom ili prezimenom radnika (filtar)
						\item Web aplikacija šalje upit bazi podatak za listu sa ili šifrom ili imenom ili prezimenom 
						\item Nakon što baza vrati listu, web aplikacija je prosljeđuje direktoru
						\item Direktor iz liste može odabrati traženog radnika i kliknuti na mjesto postavljanje plaće
						\item Šalje se uputa web aplikaciji za ažuriranje plaće radnika specifične jedinstvene šifre
						
					\end{packed_enum}
					
				\end{packed_item}
				
				\subsubsection{Dijagrami obrazaca uporabe}
					
					\textit{Prikazati odnos aktora i obrazaca uporabe odgovarajućim UML dijagramom. Nije nužno nacrtati sve na jednom dijagramu. Modelirati po razinama apstrakcije i skupovima srodnih funkcionalnosti.}
				\eject		
				
			\subsection{Sekvencijski dijagrami}
				
				\textbf{\textit{dio 1. revizije}}\\
				
				\textit{Nacrtati sekvencijske dijagrame koji modeliraju najvažnije dijelove sustava (max. 4 dijagrama). Ukoliko postoji nedoumica oko odabira, razjasniti s asistentom. Uz svaki dijagram napisati detaljni opis dijagrama.}
				\eject
	
		\section{Ostali zahtjevi}
		
			\textbf{\textit{dio 1. revizije}}\\
		 
			 \textit{Nefunkcionalni zahtjevi i zahtjevi domene primjene dopunjuju funkcionalne zahtjeve. Oni opisuju \textbf{kako se sustav treba ponašati} i koja \textbf{ograničenja} treba poštivati (performanse, korisničko iskustvo, pouzdanost, standardi kvalitete, sigurnost...). Primjeri takvih zahtjeva u Vašem projektu mogu biti: podržani jezici korisničkog sučelja, vrijeme odziva, najveći mogući podržani broj korisnika, podržane web/mobilne platforme, razina zaštite (protokoli komunikacije, kriptiranje...)... Svaki takav zahtjev potrebno je navesti u jednoj ili dvije rečenice.}
			 
			 
			 
	