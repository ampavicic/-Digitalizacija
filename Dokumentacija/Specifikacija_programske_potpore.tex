\chapter{Specifikacija programske potpore}
		
	\section{Funkcionalni zahtjevi}
			
			\textbf{\textit{dio 1. revizije}}\\
			
			\textit{Navesti \textbf{dionike} koji imaju \textbf{interes u ovom sustavu} ili  \textbf{su nositelji odgovornosti}. To su prije svega korisnici, ali i administratori sustava, naručitelji, razvojni tim.}\\
				
			\textit{Navesti \textbf{aktore} koji izravno \textbf{koriste} ili \textbf{komuniciraju sa sustavom}. Oni mogu imati inicijatorsku ulogu, tj. započinju određene procese u sustavu ili samo sudioničku ulogu, tj. obavljaju određeni posao. Za svakog aktora navesti funkcionalne zahtjeve koji se na njega odnose.}\\
			
			
			\noindent \textbf{Dionici:}
			
			\begin{packed_enum}
				
				\item Korisnici
				\item Admini sustava				
				\item Razvojni tim
				
			\end{packed_enum}
			
			\noindent \textbf{Aktori i njihovi funkcionalni zahtjevi:}
			
			
			\begin{packed_enum}
				\item  \underbar{Zaposlenik (inicijator) može:}
				
				\begin{packed_enum}
					
					\item učitavati slike
					\item skenirati dokumente
					\item pregledati povijest skeniranih dokumenata
					\item slati dokumente revizoru
					
				\end{packed_enum}
			
				\item  \underbar{Revizor (sudionik) može:}
				
				\begin{packed_enum}
					
					\item skenirati dokument
					\item provjeriti jeli dokumenat ispravan i poslati ga računovođi
					
				\end{packed_enum}
			
				\item  \underbar{Računovođa (sudionik) može:}
				
				\begin{packed_enum}
					
					\item dodjeljivati jedinstveni broj arhivu
					\item arhivirati dokument
					\item slati obavijest direktoru da se potpiše dokument
					
				\end{packed_enum}
			
				\item  \underbar{Direktor (sudionik) može:}
				
				\begin{packed_enum}
					
					\item potpisati dokumente i poslati obavijest da je dokument potpisan
					\item pregledati povijest svih dokumenata
					\item pregledati statistiku svih zaposlenika
					
				\end{packed_enum}
				
				\item  \underbar{Baza podataka (sudionik) može:}
				
				\begin{packed_enum}
					
					\item dodati nove dokumente i arhive u bazu
					\item vraćati povijest svih dokumenata
					\item vraćati statistiku svih zaposlenika
					
				\end{packed_enum}	
			
			\end{packed_enum}
			
			
			\eject 
			
			
				
			\subsection{Obrasci uporabe}
				
				\textbf{\textit{dio 1. revizije}}
				
				\subsubsection{Opis obrazaca uporabe}
					\textit{Funkcionalne zahtjeve razraditi u obliku obrazaca uporabe. Svaki obrazac je potrebno razraditi prema donjem predlošku. Ukoliko u nekom koraku može doći do odstupanja, potrebno je to odstupanje opisati i po mogućnosti ponuditi rješenje kojim bi se tijek obrasca vratio na osnovni tijek.}\\
					

					\noindent \underbar{\textbf{UC1 - Prilaganje dokumenta}}
					\begin{packed_item}
	
						\item \textbf{Glavni sudionik:} Zaposlenik
						\item  \textbf{Cilj:} Priložiti dokument
						\item  \textbf{Sudionici:} Revizor, baza podataka
						\item  \textbf{Preduvjet:} Ulogirani verificirani zaposlenik, funkcionalna kamera
						\item  \textbf{Opis osnovnog tijeka:}
						
						\item[] \begin{packed_enum}
	
							\item Zaposlenik klikće na označeno mjesto za prilaganje slika
							\item Zaposlenik označuje slike koje će se priložiti i prilaže ih
							\item Zaposlenik inicira OCR test
							\item Web aplikacija vraća dokumente u skeniranom obliku
							\item Zaposlenik označava jeli dokument krivo ili točno poslan
							\item Dokumenti se spremaju u bazu podataka (bili krivi ili točni) i šalju se na obradu revizoru
						\end{packed_enum}
						
						\item  \textbf{Opis mogućih odstupanja:}
						
						\item[] \begin{packed_item}
	
							\item[2.a] Korisnik je priložio više od 50 slika
							\item[] \begin{packed_enum}
								
								\item Javiti korisniku grešku i onemogućiti slanje
								
							\end{packed_enum}
							
						\end{packed_item}
					\end{packed_item}
				
					\noindent \underbar{\textbf{UC1 - Prilaganje dokumenta - popravljen?}}
					\begin{packed_item}
						
						\item \textbf{Glavni sudionik:} Zaposlenik, Revizor, Računovođa, Direktor
						\item  \textbf{Cilj:} Priložiti dokument
						\item  \textbf{Sudionici:} Baza podataka
						\item  \textbf{Preduvjet:} Ulogirani verificirani korisnik, funkcionalna kamera
						\item  \textbf{Opis osnovnog tijeka:}
						
						\item[] \begin{packed_enum}
							
							\item Korisnik klikće na označeno mjesto za prilaganje slika
							\item Korisnik označuje slike koje će se priložiti i prilaže ih
							\item Korisnik inicira OCR test
							\item Web aplikacija vraća dokumente u skeniranom obliku
							\item Korisnik označava jeli dokument pogrešno ili točno skeniran
							\item Dokumenti se spremaju u bazu podataka
						
						\item  \textbf{Opis mogućih odstupanja:}
						
						\item[] \begin{packed_item}
							
							\item[2.a] Korisnik je priložio više od 50 slika
							\item[] \begin{packed_enum}
								
								\item Javiti korisniku grešku i onemogućiti slanje
								
							\end{packed_enum}
							
						\end{packed_item}
					\end{packed_item}
				
				\noindent \underbar{\textbf{UC1 - Slanje dokumenta}}
				\begin{packed_item}
					
					\item \textbf{Glavni sudionik:} Zaposlenik
					\item  \textbf{Cilj:} Priložiti dokument
					\item  \textbf{Sudionici:} Revizor, baza podataka
					\item  \textbf{Preduvjet:} Ulogirani verificirani zaposlenik, funkcionalna kamera
					\item  \textbf{Opis osnovnog tijeka:}
					
					\item[] \begin{packed_enum}
						
						
						\item Zaposlenik skenira dokument
						\item Ako je dokumnet ispravno skeniran onda se šalje revizoru
					\end{packed_enum}
					
					\item  \textbf{Opis mogućih odstupanja:}
					
					\item[] \begin{packed_item}
						
						\item[2.a] Korisnik je priložio više od 50 slika
						\item[2.b] Dokument je pogrešno skreniran
						\item[] \begin{packed_enum}
							
							\item Javiti korisniku grešku i onemogućiti slanje
							
						\end{packed_enum}
						
					\end{packed_item}
				\end{packed_item}
				
					\noindent \underbar{\textbf{UC2 - Pregled povijesti skeniranih dokumenata}}
					\begin{packed_item}
						
						\item \textbf{Glavni sudionik:} Zaposlenik
						\item  \textbf{Cilj:} Dobiti listu povijsesti skeniranih dokumenata
						\item  \textbf{Sudionici:} Baza podataka
						\item  \textbf{Preduvjet:} Ulogirani verificirani zaposlenik
						\item  \textbf{Opis osnovnog tijeka:}
						
						\item[] \begin{packed_enum}
							
							\item Zaposlenik klikće na mjesto za prikaz povijesti dokumenata
							\item Web aplikacija šalje upit bazi podataka koji sadrži podatke o zaposleniku
							\item Baza podataka vraća tablicu dokumenata
							
						\end{packed_enum}
						
					\end{packed_item}
				
				\noindent \underbar{\textbf{UC3 - Verifikacija pristiglog dokumenata}}
				\begin{packed_item}
					
					\item \textbf{Glavni sudionik:} Revizor
					\item  \textbf{Cilj:} Verificirati dokumente i poslati računovođi
					\item  \textbf{Sudionici:} Računovođa
					\item  \textbf{Preduvjet:} Zaposlenik je posalo točno skenirani dokument
					\item  \textbf{Opis osnovnog tijeka:}
					
					\item[] \begin{packed_enum}
						
						\item Revizor dobiva obavijest u inboxu i poslani skenirani dokument
						\item Revizor provjerava dokument i preusmijerava ga računovođi koji je zadužen za taj tip dokumenta

					\end{packed_enum}
					
				\end{packed_item}
			
				
					
				\noindent \underbar{\textbf{UC4 - Arhiviranje dokumenata}}
				\begin{packed_item}
					
					\item \textbf{Glavni sudionik:} Računovođa
					\item  \textbf{Cilj:} (Dodjeliti jedinstveni broj arhiva)  Arhivirati dokument
					\item  \textbf{Sudionici:} Baza podataka
					\item  \textbf{Preduvjet:} Pristigli su ispravani dokumneti
					\item  \textbf{Opis osnovnog tijeka:}
					
					\item[] \begin{packed_enum}
						
					
						\item Spremi dokumente u bazu
						
					\end{packed_enum}
					
				\end{packed_item}
			 	\noindent \underbar{\textbf{UC4 - Slanje dokumenata na potpis}}
			 	\begin{packed_item}
			 		
			 		\item \textbf{Glavni sudionik:} Računovođa
			 		\item  \textbf{Cilj:} Poslati dokumnet direktoru na potpis
			 		\item  \textbf{Sudionici:} Direktor
			 		\item  \textbf{Preduvjet:} Postojanje ispravno skeniranog dokumenta
			 		\item  \textbf{Opis osnovnog tijeka:}
			 		
			 		\item[] \begin{packed_enum}
			 			
			 			
			 			\item Računovođa šalje dokument direktoru na potpis
			 			
			 			
			 		\end{packed_enum}
			 		
			 	\end{packed_item}
			
			
			
				\noindent \underbar{\textbf{UC5 - Direktorski potpis}}
				\begin{packed_item}
					
					\item \textbf{Glavni sudionik:} Direktor
					\item  \textbf{Cilj:} Potpisati dokument elektroničkim potpisom
					\item  \textbf{Sudionici:} Računovođa, baza podataka
					\item  \textbf{Preduvjet:} Računovođa mora poslati nearhivirane dokumente za potpis
					\item  \textbf{Opis osnovnog tijeka:}
					
					\item[] \begin{packed_enum}
						
						\item Direktor dobiva obavijest u inboxu i poslani nearhivirani dokument
						\item Direktor potpisuje dokument
						\item Direktor šalje obavijest da je dokument potpisan 
						\item Računovođda arhivira dokument
						
						
					\end{packed_enum}
					
				\end{packed_item}
			
				\noindent \underbar{\textbf{UC6 - Pregled potpisanih dokumenata}}
				\begin{packed_item}
					
					\item \textbf{Glavni sudionik:} Direktor
					\item  \textbf{Cilj:} Dohvatiti listu svih potpisanih dokumenata
					\item  \textbf{Sudionici:} Baza podataka
					\item  \textbf{Preduvjet:} -
					\item  \textbf{Opis osnovnog tijeka:}
					
					\item[] \begin{packed_enum}
						
						\item Direktor šalje upit bazi podataka za tablicu potpisanih dokumenata 
						\item Baza vraća tablicu potpisanih dokumenata
						
					\end{packed_enum}
				\end{packed_enum}
			
				\item  \textbf{Opis mogućih odstupanja:}
						
					\item[] \begin{packed_item}
					
					\item[2.b] U bazi nema spremljenih dokumenata
					\item[] \begin{packed_enum}
						
						\item Javiti korisniku da mora prvo spremiti dokument
						
					\end{packed_enum}
					
					\end{packed_item}
				
					
				\end{packed_item}
					
				\noindent \underbar{\textbf{UC7 - Pregled statistike}}
				\begin{packed_item}
					
					\item \textbf{Glavni sudionik:} Direktor
					\item  \textbf{Cilj:} Dohvatiti podatke o zaposlenicima
					\item  \textbf{Sudionici:} Baza podataka
					\item  \textbf{Preduvjet:} -
					\item  \textbf{Opis osnovnog tijeka:}
					
					\item[] \begin{packed_enum}
						
						\item Direktor šalje upit bazi podataka za statistiku o zaposlenicima
						\item Baza vraća tablicu zaposlenika
						
					\end{packed_enum}
					
				\end{packed_item}
				
				\subsubsection{Dijagrami obrazaca uporabe}
					
					\textit{Prikazati odnos aktora i obrazaca uporabe odgovarajućim UML dijagramom. Nije nužno nacrtati sve na jednom dijagramu. Modelirati po razinama apstrakcije i skupovima srodnih funkcionalnosti.}
				\eject		
				
			\subsection{Sekvencijski dijagrami}
				
				\textbf{\textit{dio 1. revizije}}\\
				
				\textit{Nacrtati sekvencijske dijagrame koji modeliraju najvažnije dijelove sustava (max. 4 dijagrama). Ukoliko postoji nedoumica oko odabira, razjasniti s asistentom. Uz svaki dijagram napisati detaljni opis dijagrama.}
				\eject
	
		\section{Ostali zahtjevi}
		
			\textbf{\textit{dio 1. revizije}}\\
		 
			 \textit{Nefunkcionalni zahtjevi i zahtjevi domene primjene dopunjuju funkcionalne zahtjeve. Oni opisuju \textbf{kako se sustav treba ponašati} i koja \textbf{ograničenja} treba poštivati (performanse, korisničko iskustvo, pouzdanost, standardi kvalitete, sigurnost...). Primjeri takvih zahtjeva u Vašem projektu mogu biti: podržani jezici korisničkog sučelja, vrijeme odziva, najveći mogući podržani broj korisnika, podržane web/mobilne platforme, razina zaštite (protokoli komunikacije, kriptiranje...)... Svaki takav zahtjev potrebno je navesti u jednoj ili dvije rečenice.}
			 
			 
			 
	