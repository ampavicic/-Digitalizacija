\chapter{Arhitektura i dizajn sustava}
		
		\textbf{\textit{dio 1. revizije}}\\

		\textit{ Potrebno je opisati stil arhitekture te identificirati: podsustave, preslikavanje na radnu platformu, spremišta podataka, mrežne protokole, globalni upravljački tok i sklopovsko-programske zahtjeve. Po točkama razraditi i popratiti odgovarajućim skicama:}
	\begin{itemize}
		\item 	\textit{izbor arhitekture temeljem principa oblikovanja pokazanih na predavanjima (objasniti zašto ste baš odabrali takvu arhitekturu)}
		\item 	\textit{organizaciju sustava s najviše razine apstrakcije (npr. klijent-poslužitelj, baza podataka, datotečni sustav, grafičko sučelje)}
		\item 	\textit{organizaciju aplikacije (npr. slojevi frontend i backend, MVC arhitektura) }		
	\end{itemize}

	
		

		

				
		\section{Baza podataka}
			
			\textbf{\textit{dio 1. revizije}}\\
			
		\textit{Potrebno je opisati koju vrstu i implementaciju baze podataka ste odabrali, glavne komponente od kojih se sastoji i slično.}
		\text{U našoj web aplikaciji koristili smo relacijsku bazu podataka. Baza se sastoji od tablica koje imaju svoje atribute. Baza podataka je jako bitna da bi web aplikacija radila jer su na njoj pohranjeni svi podaci o zaposlenicima i svi skenirani dokumenti.}
		\newline
		\text{Entiteti baze podataka su sljedeći:}
		\begin{packed_item}
			\item {Dokument}
			\item {Zaposlenik}
			\item {Korisnički račun}
			\item {Uloga}
			\item {Racun}
			\item {Ponuda}
			\item {Interni}
		\end{packed_item}
	
		
			\subsection{Opis tablica}
			

				\textit{Svaku tablicu je potrebno opisati po zadanom predlošku. Lijevo se nalazi točno ime varijable u bazi podataka, u sredini se nalazi tip podataka, a desno se nalazi opis varijable. Svjetlozelenom bojom označite primarni ključ. Svjetlo plavom označite strani ključ}
				
				\textbf{Zaposlenik}  Ovaj entitet sadrži osnovne informacije o zaposleniku. Sadrži sljedeće atribute: genID, PID, Name, Surname, Residence, Salary i roleID. genID je alternativni ključ dok je PID primarni ključ. Entitet je povezan s entitetom \textbf{UserAccount} preko genID i \textbf{Role} preko roleID.
				
				
				\begin{longtblr}[
					label=none,
					entry=none
					]{
						width = \textwidth,
						colspec={|X[6,l]|X[6, l]|X[20, l]|}, 
						rowhead = 1,
					} %definicija širine tablice, širine stupaca, poravnanje i broja redaka naslova tablice
					\hline \multicolumn{3}{|c|}{\textbf{Employee}}	 \\ \hline[3pt]
					\SetCell{LightGreen}genID & VARCHAR	&  kod koji zaposlenik dobije od direktora 	\\ \hline
					PID & VARCHAR	&  OIB zaposlenika 	\\ \hline
					Name	& VARCHAR &   Ime zaposlenika	\\ \hline 
					Surname & VARCHAR & Prezime zaposlenika \\ \hline
					Residence & VARCHAR &  Mjesto stanovanja \\ \hline 
					Salary & INT	& Plaća zaposlenika 		\\ \hline 
					\SetCell{LightBlue} roleID	& VARCHAR &  jedinstveni kod pozicije (uloge) 	\\ \hline 
				\end{longtblr}
				
				
				\textbf{Role}  Ovaj entitet sadržava roleID te RoleName što su ujedno i atributi. Povezan je sa entitetom \textbf{Employee} preko roleID.
				
				\begin{longtblr}[
					label=none,
					entry=none
					]{
						width = \textwidth,
						colspec={|X[6,l]|X[6, l]|X[20, l]|}, 
						rowhead = 1,
					} %definicija širine tablice, širine stupaca, poravnanje i broja redaka naslova tablice
					\hline \multicolumn{3}{|c|}{\textbf{Role}}	 \\ \hline[3pt]
					\SetCell{LightGreen}roleID & INT	& jedinstveni kod uloge 	\\ \hline
					RoleName	& VARCHAR &  ime uloge	\\ \hline 
				\end{longtblr}
			
			\textbf{Document} Ovaj entitet sadrži osnovne informacije o dokumentu. Sadrži sljedeće atribute: documentID, arhiviran, potpis te korisničko ime. Entitet je nadklasa entitetima \textbf{Receipt}, \textbf{Intern} i \textbf{Offer} te je s tim entitetima povezan preko svog documentID-a. Isto tako, povezan je s entitetom \textbf{UserAccount} preko atributa Username.
			
				\begin{longtblr}[
					label=none,
					entry=none
					]{
						width = \textwidth,
						colspec={|X[6,l]|X[6, l]|X[20, l]|}, 
						rowhead = 1,
					} %definicija širine tablice, širine stupaca, poravnanje i broja redaka naslova tablice
					\hline \multicolumn{3}{|c|}{\textbf{Document}}	 \\ \hline[3pt]
					\SetCell{LightGreen}documentID & INT	& jedinstveni kod dokumenta 	\\ \hline
					Archived	& INT &  je li dokument arhiviran (da/ne))	\\ \hline 
					Signature & INT & je li potreban direktorski potpis (da/ne) \\ \hline
					\SetCell{LightBlue} Username & VARCHAR & korisnicko ime osobe koja je skenirala dokument \\ \hline
				\end{longtblr}
			
			\textbf{Intern} Ovaj entitet je vrsta dokumenta, dakle to je specijalzacija entiteta \textbf{Document}. Sadrži atribute INT4 što je alternativni ključ, documentID što je primarni ključ i ClientName.
				\begin{longtblr}[
					label=none,
					entry=none
					]{
						width = \textwidth,
						colspec={|X[6,l]|X[6, l]|X[20, l]|}, 
						rowhead = 1,
					} %definicija širine tablice, širine stupaca, poravnanje i broja redaka naslova tablice
					\hline \multicolumn{3}{|c|}{\textbf{Intern}}	 \\ \hline[3pt]
					\SetCell{LightGreen}documentID & INT	& jedinstveni kod uloge 	\\ \hline
					INT4	& VARCHAR &  "INT" + 4 znamenke	\\ \hline
					ClientName & VARCHAR & ime klijenta \\ \hline
				\end{longtblr}		
			
			\textbf{Receipt} Ovaj entitet je vrsta dokumenta, dakle to je specijalzacija entiteta \textbf{Document}. Sadrži atribute R6 što je alternativni ključ i documentID što je primarni ključ.
				\begin{longtblr}[
					label=none,
					entry=none
					]{
						width = \textwidth,
						colspec={|X[6,l]|X[6, l]|X[20, l]|}, 
						rowhead = 1,
					} %definicija širine tablice, širine stupaca, poravnanje i broja redaka naslova tablice
					\hline \multicolumn{3}{|c|}{\textbf{Receipt}}	 \\ \hline[3pt]
					\SetCell{LightGreen}documentID & INT	& jedinstveni kod dokumenta 	\\ \hline
					R6	& VARCHAR &  "R" + 6 znamenki	\\ \hline
				\end{longtblr}	
			
			\textbf{Offer}	Ovaj entitet je vrsta dokumenta, dakle to je specijalzacija entiteta \textbf{Document}. Sadrži atribute P9 što je alternativni ključ te documentID što je primarni ključ.	
				\begin{longtblr}[
					label=none,
					entry=none
					]{
						width = \textwidth,
						colspec={|X[6,l]|X[6, l]|X[20, l]|}, 
						rowhead = 1,
					} %definicija širine tablice, širine stupaca, poravnanje i broja redaka naslova tablice
					\hline \multicolumn{3}{|c|}{\textbf{Offer}}	 \\ \hline[3pt]
					\SetCell{LightGreen}documentID & INT	& jedinstveni kod uloge 	\\ \hline
					P9	& VARCHAR &  "P" + 9 znamenki	\\ \hline
				\end{longtblr}
			
			\textbf{UserAccount} Ovaj entitet sadrži podatke o korisnčkim računima. Sadrži sljedeće atribute: Username što je ujedno i primarni key, password, email, enabled i PID što je strani ključ. Entitet je povezan s entitetom \textbf{Document} preko atributa Username i s entitetom \textbf{Employee} preko atributa PID.
				
				\begin{longtblr}[
					label=none,
					entry=none
					]{
						width = \textwidth,
						colspec={|X[6,l]|X[6, l]|X[20, l]|}, 
						rowhead = 1,
					} %definicija širine tablice, širine stupaca, poravnanje i broja redaka naslova tablice
					\hline \multicolumn{3}{|c|}{\textbf{UserAccount}}	 \\ \hline[3pt]
					\SetCell{LightGreen}Username & VARCHAR	& korisničko ime	\\ \hline
					password	& VARCHAR &  šifra	\\ \hline 
					email & VARCHAR & mail adresa \\ \hline
					\SetCell{LightBlue} PID & VARCHAR & oib zaposlenika \\ \hline
				\end{longtblr}
				
			
			
			
			
			
			
			
			
			
				
			
			\subsection{Dijagram baze podataka}
				\textit{ U ovom potpoglavlju potrebno je umetnuti dijagram baze podataka. Primarni i strani ključevi moraju biti označeni, a tablice povezane. Bazu podataka je potrebno normalizirati. Podsjetite se kolegija "Baze podataka".}
			
			\eject
			
			
		\section{Dijagram razreda}
		
			\textit{Potrebno je priložiti dijagram razreda s pripadajućim opisom. Zbog preglednosti je moguće dijagram razlomiti na više njih, ali moraju biti grupirani prema sličnim razinama apstrakcije i srodnim funkcionalnostima.}\\
			
			\textbf{\textit{dio 1. revizije}}\\
			
			\textit{Prilikom prve predaje projekta, potrebno je priložiti potpuno razrađen dijagram razreda vezan uz \textbf{generičku funkcionalnost} sustava. Ostale funkcionalnosti trebaju biti idejno razrađene u dijagramu sa sljedećim komponentama: nazivi razreda, nazivi metoda i vrste pristupa metodama (npr. javni, zaštićeni), nazivi atributa razreda, veze i odnosi između razreda.}\\
			
			\textbf{\textit{dio 2. revizije}}\\			
			
			\textit{Prilikom druge predaje projekta dijagram razreda i opisi moraju odgovarati stvarnom stanju implementacije}
			
			
			
			\eject
		
		\section{Dijagram stanja}
			
			
			\textbf{\textit{dio 2. revizije}}\\
			
			\textit{Potrebno je priložiti dijagram stanja i opisati ga. Dovoljan je jedan dijagram stanja koji prikazuje \textbf{značajan dio funkcionalnosti} sustava. Na primjer, stanja korisničkog sučelja i tijek korištenja neke ključne funkcionalnosti jesu značajan dio sustava, a registracija i prijava nisu. }
			
			
			\eject 
		
		\section{Dijagram aktivnosti}
			
			\textbf{\textit{dio 2. revizije}}\\
			
			 \textit{Potrebno je priložiti dijagram aktivnosti s pripadajućim opisom. Dijagram aktivnosti treba prikazivati značajan dio sustava.}
			
			\eject
		\section{Dijagram komponenti}
		
			\textbf{\textit{dio 2. revizije}}\\
		
			 \textit{Potrebno je priložiti dijagram komponenti s pripadajućim opisom. Dijagram komponenti treba prikazivati strukturu cijele aplikacije.}